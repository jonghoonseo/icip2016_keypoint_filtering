%!TEX root = ../OpenNUI_Platform__Design_of_Smart_Space_Interaction_Platform.tex

\subsection{Summary}

%본 논문에서는 Smart Space에서의 Seamless한 정보의 제공을 위하여 marker-less AR 기술을 개발하였다. Smart Space 환경에서는 Computing power가 제한적이기 때문에 이를 극복하기 위하여 경량화 된 marker-less AR 기술이 필요하다. 따라서 본 논문에서는 오프라인 학습 단계에서 detect된 특징점들을 평가하여 좋은 특징점만을 이용하여 학습하는 특징점 필터링 알고리즘을 제안하였다. 좋은 특징점은 영상의 변화에도 강인하게 detect되고, 자기 자신과의 매치 유사도가 높고, 다른 특징점과의 매치 유사도는 낮은 특징점을 의미한다. 이러한 특징을 반영하여 특징점을 filter 할 수 있는 score function을 정의하였다. 이를 통하여 offline train 단계에서 검출된 특징점을 filter 하여 저장함으로써, 전체적인 매치 연산을 줄여 속도를 향상시키면서도 영상의 인식율과 keypoint의 precision 을 높일 수 있었다.

In this dissertation, marker-less AR technology was developed for the purpose of providing seamless information in a smart space. In a smart space environment, the computing power is limited, so to overcome this limitation, lightweight marker-less AR technology is required. Thus, in this dissertation, keypoints filtering algorithm, which used only the good keypoints selected after the evaluation the keypoints detected in the offline training phase, was proposed. Good keypoints are rigidly detected despite the change of images; show high match similarity with themselves; and show low match similarity with other keypoints. The score function enabling it to filter the keypoints by reflecting above features was defined. It was possible to filter and save the keypoints detected in the offline training phase, which in turn increased the speed by reducing the overall match computation and thereby increased the precision of keypoints and the image recognition rate.  
