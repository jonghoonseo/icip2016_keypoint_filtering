%!TEX root = ../icip_jseo.tex
% -*- root: ../icip_jseo.tex -*-

\section{Conclusion}

%본 논문에서는 Smart Space에서의 Seamless한 정보의 제공을 위하여 marker-less AR 기술을 개발하였다. Smart Space 환경에서는 Computing power가 제한적이기 때문에 이를 극복하기 위하여 경량화 된 marker-less AR 기술이 필요하다. 따라서 본 논문에서는 오프라인 학습 단계에서 detect된 특징점들을 평가하여 좋은 특징점만을 이용하여 학습하는 특징점 필터링 알고리즘을 제안하였다. 좋은 특징점은 영상의 변화에도 강인하게 detect되고, 자기 자신과의 매치 유사도가 높고, 다른 특징점과의 매치 유사도는 낮은 특징점을 의미한다. 이러한 특징을 반영하여 특징점을 filter 할 수 있는 score function을 정의하였다. 이를 통하여 offline train 단계에서 검출된 특징점을 filter 하여 저장함으로써, 전체적인 매치 연산을 줄여 속도를 향상시키면서도 영상의 인식율과 keypoint의 precision 을 높일 수 있었다.

% 로컬 키포인트 기반의 이미지 매칭 방법은 영상에서 특징이 되는 점(salient points)들의 특징을 저장하고 이를 실시간 영상과 매칭한다. 하지만,



In this paper, we propose a keypoint filtering approach to accomplish not only fast, but also robust and more reliable matching for mobile computing environment. Conventional keypoint matching methods rarely consider about quality of stored keypoint database. So, to accomplish robust matching quality, they require more keypoints. However, those redundant keypoints degrade matching quality. To overcome this problem, we evaluate detected keypoints and store only filtered keypoints. These filtered keypoints are repeatedly detected despite the change of images; show high match similarity with identical keypoints; and show low match similarity with other keypoints. Experimental results show that our filtering approach is effective in terms of both matching speed and quality thus applying this approach feasible for mobile image matching applications such as mobile object recognition or aumented reality.

% As a future work, we plan to apply this approach to other mobile applications and 

% 이 논문에서는 모바일 컴퓨팅 환경에서의 영상 매칭을 위하여 빠른 속도 뿐만이 아니라, 높은 인식율을 제공하는 새로운 키포인트 필터링 방법을 제안하였다. 기존의 키포인트 기반 매칭 방법은 키포인트의 품질에 대한 고려가 이루어지지 않고,반복이 잘 되는 점들을 저장하였다. 이에 따라서, 정확한 매칭 결과를 얻기 위해서는 많은 수의 키포인트가 필요하여 속도가 떨어질 뿐만 아니라, 질나쁜 키포인트로 인하여 false 매칭의 확률도 높았다. 본 논문에서는 이러한 방법을 극복하기 위하여 offline learning/training 단계에서 추출된 키포인트를 평가하여 매칭 성능이 높은 점들만 저장하는 방법을 제안한다. 







% In this paper, keypoints filtering algorithm, which used only the good keypoints selected after the evaluation the keypoints detected in the offline training phase, was proposed. 

% Good keypoints are rigidly detected despite the change of images; show high match similarity with themselves; and show low match similarity with other keypoints. The score function enabling it to filter the keypoints by reflecting above features was defined. It was possible to filter and save the keypoints detected in the offline training phase, which in turn increased the speed by reducing the overall match computation and thereby increased the precision of keypoints and the image recognition rate.  
