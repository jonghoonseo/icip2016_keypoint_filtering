%!TEX root = ../OpenNUI_Platform__Design_of_Smart_Space_Interaction_Platform.tex


\subsection{Keypoint Descriptor Filtering Algorithm}

feature matching 기반의 증강현실 시스템에서는 online process 이전에 reference images들을 오프라인으로 학습(train)하여 비교 대상인 feature database 를 생성하게 된다. 특히 앞의 matching data structure 연구에서 보듯이 온라인 인식의 성능을 높이기 위하여 오프라인 처리에서 다양한 연산을 적용하여 매칭 구조체를 구성함으로써 온라인 인식의 속도나 인식율을 높이기 위한 방법도 제안되고 있다. 

이러한 기존의 특징점 매칭 방법들은 Feature Detection 알고리즘에서 검출한 특징점을 단순히 모두 학습에 사용하였다. 하지만, 특징점 검출 알고리즘은 Description 알고리즘과 독립적으로 수행되기 때문에 검출(Detect)된 특징점들에 대한 Descriptor가 매칭 성능을 보장할 수 없다는 문제가 있다. 

따라서 본 논문에서는 오프라인 학습 과정에서 학습 대상 특징점을 평가하고, 이를 통하여 실시간 인식에 강인한 성능을 제공하는 특징점만을 선정하였다. 이러한 특징점들만을 데이터베이스에 저장함으로써 인식 품질을 유지하면서도 인식 속도를 향상시킬 수 있는 방법을 제안한다.

\subsubsection{Definition of Good Keypoins}

제안하는 방법은 detect된 특징점을 분석하여 인식에 좋은 정도를 측정함으로써 좋은 특징점만을 filtering 한다. 이를 위하여 좋은 특징점을 정의하였다.

첫 번째 인식에 좋은 특징점의 조건은 대상 영상이 다양하게 변화되는 환경에서도 안정적으로 특징점으로 검출되어야 한다는 점이다. 실제 온라인 매칭 과정에서 입력되는 카메라 영상에는 인식하고자 하는 대상 영상이 회전이나 크기, perspective, 노이즈, 조명 등의 다양한 형태의 변환이 적용되어 있다. 인식에 좋은 특징점은 이러한 변환된 영상에서도 안정적으로 특징점이 검출이 됨으로써 Descriptor를 생성할 수 있도록 하는 점이어야 한다.

이러한 안정적 특징점 검출은 Repeatability Score로 측정이 가능하다. Repeatability는 아래 그림과 같이 전체 변환된 영상의 개수 대비 변환된 영상에서 키포인트가 변환되어 존재하는 경우의 비율로 계산된다.
\begin{equation}
p_{repeatability}(p_i) = \frac{n_i^{overlap}}{N}
\end{equation}

여기에서 $n_i^{overlap}$는 변환된 영상($T_t(I)$)의 키포인트 집합($K_t'$)에 키포인트($p_i$)가 변환된 점이 존재($T(p_i)\in K_t'$)하는 횟수로 계산되며, $N$은 총 변환된 영상의 개수로 모든 키포인트가 동일한 값을 가지게 된다.

두 번째 인식에 좋은 특징점 집합의 조건은 해당 키포인트가 변화된 영상에서 만들어진 descriptor와의 매치가 실패가 적어야 한다(Similarity)는 점이다.

이는 해당 특징점의 Sensitivity(True Positive Rate)와 연관이 있다. Reference 영상의 어떤 특징점($p_i$)에 대하여 학습 과정에서 다양하게 변환된 영상($T_t(I)$)에서의 모든 키포인트 집합($K_t'$)의 Descriptor 사이의 매칭을 계산하여 해당 특징점에 대한 Genuine 분포와 Impostor 분포를 계산할 수 있다. 이 때, 해당 키포인트가 변화된 영상에서의 Descriptor와의 매치가 실패가 적기 위해서는, Genuine 분포가 매치 임계 거리(Match Distance Threshold)에서부터 충분히 떨어져 작은 값을 가져야 한다. 이를 위하여 Genuine 분포의 평균을 이용하여 이를 측정하였다. 식 \ref{eq:similarity}에서 보는 것과 같이, Genuine 분포가 작을 수록 좋은 특징점이기 때문에 genuine 분포의 평균을 normalize 하여 1에서 뺌으로써 평가 함수를 계산하였다.
\begin{equation}
p_{similarity}(p_i) = 1 - \frac{\mu_{gen,i} - \min_i \mu_{gen,i}}{\max_i \mu_{gen, i} - \min_i \mu_{gen,i}}
\end{equation}	\label{eq:similarity}

세 번째 조건은 학습된 특징점이 자신과 다른 특징점과는 매치가 이루어지지 않아야 한다(Separability)는 점이다. 이는 각 특징점의 Impostor 분포와 관련이 있다. 다양하게 변화된 영상에서 추출된 특징점 중 자기 자신이 변환된 특징점이 아닌 다른 특징점들과의 매칭 분포를 Impostor 분포라 한다. 어떤 특징점이 자신 이외의 다른 키포인트와 낮은 매칭 성공률을 보여주기 위해서는 Genuine 분포와 Impostor 분포가 잘 구분이 되어야 한다.

이를 위하여 본 논문에서는 Fisher's Discriminant Ratio\cite{fisher_use_1936}를 이용하였다. 이는 1-dimensional, two class problem에서 sample의 평균과 분산에 의하여 두 클래스 사이의 거리를 측정하는 방식이다. 두 번째 Similarity 조건에 의하여 Genuine 분포가 충분히 작음이 보장이 되었기 때문에 Impostor의 분포가 Genuine 분포에 비하여 충분히 떨어져 있다면 Impostor 분포에 존재하는 특징점들과는 매치가 이루어지지 않음이 보장된다. separability 값 역시 수식 \ref{eq:norm_separability}과 같이 normalize 과정이 필요하다.

\begin{equation}
FDR(p_i) = \frac{(\mu_{gen,i} - \mu_{imp, i})^2}{\sigma_{gen, i}^2 + \sigma_{imp, i}^2}
\end{equation}
\begin{equation}
p_{separability}(p_i) = \frac{FDR(p_i) - \min_i {FDR(p_i)}}{\max_{i} {s_i}}
\end{equation}	\label{eq:norm_separability}
이렇게 계산된 세가지 criteria를 이용하여 각 특징점의 score function을 정의할 수 있다. 세 조건은 종속이므로 식 \ref{eq:score_function}와 같이 정의된다.
\begin{equation}
gf(p_i) = p_{repeatability}(p_i)p_{similarity}(p_i)p_{separability}(p_i)
\end{equation} \label{eq:score_function}



% \subsubsection{}
%%%%% 시간이 되면 실험 부분을 추가하자 %%%%%

