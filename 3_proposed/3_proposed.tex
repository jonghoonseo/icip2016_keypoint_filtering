%!TEX root = ../icip_jseo.tex
% -*- root: ../icip_jseo.tex -*-

\section{Proposed Method}
\label{sec:proposed}


%제안하는 방법은 detect된 특징점을 분석하여 인식에 좋은 정도를 측정함으로써 좋은 특징점만을 filtering 한다. 이를 위하여 좋은 특징점을 정의하였다.

The proposed filtering method stores only good keypoints by analyzing the characteristics of detected keypoints and measuring the degree of effectiveness for image matching. 

There are several factors which good keypoints should follows:

%첫 번째 인식에 좋은 특징점의 조건은 대상 영상이 다양하게 변화되는 환경에서도 안정적으로 특징점으로 검출되어야 한다는 점이다. 실제 온라인 매칭 과정에서 입력되는 카메라 영상에는 인식하고자 하는 대상 영상이 회전이나 크기, perspective, 노이즈, 조명 등의 다양한 형태의 변환이 적용되어 있다. 인식에 좋은 특징점은 이러한 변환된 영상에서도 안정적으로 특징점이 검출이 됨으로써 Descriptor를 생성할 수 있도록 하는 점이어야 한다.
\textit{Repeatability}: Good keypoints need to be stably detected even in various environment. In actual matching environments, a wide range of transformed image degradation may occurred, such as the rotation, size, noise and lighting of the targeted images. The good keypoints have to be stably extracted against those transformation.

%이러한 안정적 특징점 검출은 Repeatability Score로 측정이 가능하다. Repeatability는 아래 그림과 같이 전체 변환된 영상의 개수 대비 변환된 영상에서 키포인트가 변환되어 존재하는 경우의 비율로 계산된다.
The detection of stable keypoints can be measured by \textit{Repeatability} condition. \textit{Repeatability} is calculated by the ratio between the total number of synthesized images and the number of cases where the transformed keypoints are existent in the synthesized images.


\begin{equation}
p_{rep}(p_i) = \frac{n_i^{overlap}}{N}
\end{equation}

% 여기에서 $n_i^{overlap}$는 변환된 영상($T_t(I)$)의 키포인트 집합($K_t'$)에 키포인트($p_i$)가 변환된 점이 존재($T(p_i)\in K_t'$)하는 횟수로 계산되며, $N$은 총 변환된 영상의 개수로 모든 키포인트가 동일한 값을 가지게 된다.

\noindent
where $p_{rep}(p_i)$ represents repeatability score of given point $p_i$, $n_i^{overlap}$ is calculated by the frequency of the existence of transformed keypoint($p_i$) in the set of keypoints($T(p_i)\in K_t'$) of synthesized images $T_t(I)$; $N$ is the total number of synthesized images ; and all keypoints have single value. 

%두 번째 인식에 좋은 특징점 집합의 조건은 해당 키포인트가 변화된 영상에서 만들어진 descriptor와의 매치가 실패가 적어야 한다(Similarity)는 점이다.
%이는 해당 특징점의 Sensitivity(True Positive Rate)와 연관이 있다. Reference 영상의 어떤 특징점($p_i$)에 대하여 학습 과정에서 다양하게 변환된 영상($T_t(I)$)에서의 모든 키포인트 집합($K_t'$)의 Descriptor 사이의 매칭을 계산하여 해당 특징점에 대한 Genuine 분포와 Impostor 분포를 계산할 수 있다. 이 때, 해당 키포인트가 변화된 영상에서의 Descriptor와의 매치가 실패가 적기 위해서는, Genuine 분포가 매치 임계 거리(Match Distance Threshold)에서부터 충분히 떨어져 작은 값을 가져야 한다. 이를 위하여 Genuine 분포의 평균을 이용하여 이를 측정하였다. 식 \ref{eq:similarity}에서 보는 것과 같이, Genuine 분포가 작을 수록 좋은 특징점이기 때문에 genuine 분포의 평균을 normalize 하여 1에서 뺌으로써 평가 함수를 계산하였다.

\textit{Similarity}: Good keypoints need to be well-matched with identical keypoints even though targeted images change in various ways. With regard to a certain keypoint($p_i$) of reference images, genuine distribution' and imposter distribution' for the corresponding keypoint can be measured by calculating the matching between the descriptors of all the sets of keypoints($p_i$) in images($T_t(I)$) transformed in various ways  during the training process. At this time, to reduce the failure in matching the corresponding keypoints and the descriptors in the transformed images, the genuine distribution needs to have small value, being far enough away from match distance threshold. To this effect, it was measured using the mean of genuine distribution. As shown in Equation \eqref{eq:similarity}, the keypoints with the decreasing the genuine distribution are better, so the evaluation function was calculated by normalizing the mean of the genuine distribution and subtracting its value from 1.  

\begin{equation} \label{eq:similarity}
p_{sim}(p_i) = 1 - \frac{\mu_{gen,i} - \min_i \mu_{gen,i}}{\max_i \mu_{gen, i} - \min_i \mu_{gen,i}}
\end{equation}	
where $p_{sim}(p_i)$ represents similarity score of given point $p_i$, $\mu_{gen, i}$ represents mean of genuine distribution of give point $p_i$. 

%세 번째 조건은 학습된 특징점이 자신과 다른 특징점과는 매치가 이루어지지 않아야 한다(Separability)는 점이다. 이는 각 특징점의 Impostor 분포와 관련이 있다. 다양하게 변화된 영상에서 추출된 특징점 중 자기 자신이 변환된 특징점이 아닌 다른 특징점들과의 매칭 분포를 Impostor 분포라 한다. 어떤 특징점이 자신 이외의 다른 키포인트와 낮은 매칭 성공률을 보여주기 위해서는 Genuine 분포와 Impostor 분포가 잘 구분이 되어야 한다.
%이를 위하여 본 논문에서는 Fisher's Discriminant Ratio\cite{fisher_use_1936}를 이용하였다. 이는 1-dimensional, two class problem에서 sample의 평균과 분산에 의하여 두 클래스 사이의 거리를 측정하는 방식이다. 두 번째 Similarity 조건에 의하여 Genuine 분포가 충분히 작음이 보장이 되었기 때문에 Impostor의 분포가 Genuine 분포에 비하여 충분히 떨어져 있다면 Impostor 분포에 존재하는 특징점들과는 매치가 이루어지지 않음이 보장된다. separability 값 역시 수식 \ref{eq:norm_separability}과 같이 normalize 과정이 필요하다.

\textit{Separability}: The trained keypoints and other keypoints shall not be matched, which is associated with the imposter distribution of each keypoint. Of the keypoints extracted from the images converted in various images, the distribution of the matching with other keypoints rather than the converted keypoints themselves are referred to as imposter distribution. Thus for a specific keypoint to show the low success rate of matching with other keypoints rather than themselves, in is necessary that the genuine distribution and imposter distribution are well classified. To this effect, in this paper, \textit{Fisher's Discriminant Ratio\cite{fisher_use_1936}} was used. It measures the distance between two classes by the mean and distribution of sample in 1-dimensional, two class problems. Since the second Similarity condition ensures the genuine distribution is small enough, the nonexistence of the matching with the keypoints in the imposter distribution is ensured if the importer distribution is far enough away compared with the genuine distribution. Separability value also requires the normalization process as shown in equation \eqref{eq:norm_separability}.  


\begin{equation}
FDR(p_i) = \frac{(\mu_{gen,i} - \mu_{imp, i})^2}{\sigma_{gen, i}^2 + \sigma_{imp, i}^2}
\end{equation}
\begin{equation}\label{eq:norm_separability}
p_{sep}(p_i) = \frac{FDR(p_i) - \min_i {FDR(p_i)}}{\max_{i} {s_i}}
\end{equation}	
where $p_{sep}(p_i)$ represents similarity score of given point $p_i$, $\mu_{gen, i}, \mu_{imp, i}$ represent mean of genuine and impostor distribution of give point $p_i$, respectively. $\sigma_{gen,i}, \sigma_{imp, i}$ represent standard deviation of genuine and impostor distribution, respectively.


%이렇게 계산된 세가지 criteria를 이용하여 각 특징점의 score function을 정의할 수 있다. 세 조건은 종속이므로 식 \ref{eq:score_function}와 같이 정의된다.

The score functions of each keypoint can be defined using 3 criteria calculated as above. The 3 conditions are dependent, so can be defined as shown in Equation \eqref{eq:score_function}. 

\begin{equation}\label{eq:score_function}
gf(p_i) = p_{rep}(p_i)p_{sim}(p_i)p_{sep}(p_i)
\end{equation} 



% \subsubsection{}
%%%%% 시간이 되면 실험 부분을 추가하자 %%%%%

