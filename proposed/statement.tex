%!TEX root = ../icip_jseo.tex
% -*- root: ../icip_jseo.tex -*-

\subsection{Problem}
In general, keypoint matching methods 
일반적으로 키포인트 기반의 매칭 방법은 미리 학습된 키포인트 데이터베이스 $K^R$와 입력된 영상을 분석하여 생성된 키포인트 집합 $K^I$를 비교하여, 가장 유사한 키포인트 pair 집합  $C = \{(k_i^R, k_j^I) | \argmin\limits_{k_i\in K^R}  \argmin\limits_{k_j\in K^I} | k_i^R - k_j^I | \}$ 을 계산하는 과정이다. 기존의 키포인트 매칭 방법은 검출된 키포인트 집합 $K^R$을 그대로 사용하였으나, 본 논문에서는 키포인트 평가 함수($s(k)$)를 제안하여 이러한 평가 함수에 의하여 필터링된 집합 $K' = \{ k | s(k) is\;\; high \} \in K^R$ 을 계산하고, 이러한 필터링 된 부분집합 $K'$는 필터링 되지 않은 $K^R$에 비하여 더 높은 인식성능을 보여줌을 증명하고자 한다. \texttt{조금만 더 늘여쓰자}